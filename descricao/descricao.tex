\documentclass[10pt,a4paper,titlepage]{hitec}

\usepackage[brazil]{babel}
\usepackage[utf8x]{inputenc}
\usepackage[T1]{fontenc}
\usepackage{kpfonts} % load a font with all the characters

\usepackage{hyperref}
\usepackage{amsthm}
\usepackage{amssymb}
\usepackage{bm}
\usepackage{graphicx,url}


\author{Silvana Trindade e Maurício André Cinelli}
\title{CONJUNTO FIRST E FOLLOW }

\begin{document}
\maketitle

\section*{Analisador Preditivo Tabular}
A análise preditiva tabular implementa um autômato com pilha controlado por uma tabela de análise. 
O princípio do reconhecimento preditivo é a determinação da produção a ser aplicada, cujo lado direito irá substituir o símbolo não terminal que se encontra no topo da pilha. 
O analisador busca a produção a ser aplicada na tabela de análise, levando em conta o não terminal no topo da pilha e o token sob o cabeçote de leitura.

A fita de entrada contém a sentença a ser analisada seguida de \$, símbolo que marca o fim da sentença.
Inicialmente, a pilha contém \$, que marca a sua base, seguido do símbolo inicial da gramática.
A tabela de análise á uma matriz com $n$ linhas e $(t + 1)$ colunas, onde $n$ é o número de símbolos não-terminais e $t$ é o número de símbolos terminais.

Considere $X$ o símbolo no topo da pilha e $a$ o terminal da fita.
O analisador executa uma das três possíveis ações:

\begin{enumerate}
\item se $X = a = $ \$, o analisador para, aceitando a senteça
\item se $X = a \neq$  \$, o analisador desempilha a e avança o cabeçote de leitura para o próximo símbolo na fita.
\item se $X$ não terminal, o analisador consulta a entrada $M[X,a]$ da tabela de análise.
 Então o $X$ é substituído pelo lado direito da regra de produção escolhida.
\end{enumerate}

\subsection*{Conjuntos First e Follow}
Dada uma palavra $\gamma$ contendo símbolos não terminais e terminais, $FIRST(\gamma)$ é o conjunto de todos os símbolos terminais que podem começar cqualquer palavra derivada de $\gamma$ .

Se duas produções $X \rightarrow \gamma_1$ e $X \rightarrow \gamma_2$ possuem o mesmo símbolo não terminal $X$ à esquerda da produção e seus símbolos à direita possuem conjuntos $FIRST$ sobrepostos, então esta gramática não pode ser analisada utilizando algoritmo dscendente recursivo. 
Se um símbolo terminal $I$ pertence a $FIRST(\gamma_1)$ e à $FIRST(\gamma_2)$, então a função que representa a produção $X$ em um analisador descendente recursivo não saberá qual cláusula condicional executar quando o token de entrada é $I$.
O calculo do conjunto $FIRST$ aparentemente é muto simples: se $\gamma = XYZ$, aparentemente $Y$ e $Z$ podem ser ignorados, e $FIRST(X)$ é a unica coisa que realmente interessa para o cálculo de $FIRST(\gamma)$.
Infelizmente as coisas não são tão simples assim, como por exemplo na gramática apresentada abaixo:
\\
\begin{center}
\begin{tabular}{lrc}
$Z \rightarrow d$ & $Y \rightarrow \varepsilon$\\
$Z \rightarrow XYZ$ & $X \rightarrow Y $\\
$Y \rightarrow c$ & $X \rightarrow a$\\
\end{tabular}
\end{center}

Nesta gramática $Y$ pode produzir a palavra vazia - $\varepsilon$ portanto $X$ pode produzir a palavra vazia - $\varepsilon$, descobre-se que $FIRST(XYZ)$ deve incluir $FIRST(Z)$.

Portanto, no cálculo dos conjuntos FIRST, é necessário armazenar quais símbolos poderão produzir palavras vazias; tais símbolos são denominados \textit{nullable}.
É necessário armazenar também o que pode seguir um símbolo \textit{nullable}.

Dada uma cadeia particular \textit{w} de uma gramática contendo símbolos terminais e não terminais.

\begin{itemize}
\item \textit{nullable}(X) é verdadeiro se $X$ pode derivar a palavra vazia
\item $FIRST(w)$ é o conjunto de terminais que podem iniciar palavras derivadas de $\gamma$
\item $FOLLOW(X)$ é o conjunto de terminais que podem imediatamente seguir $X$. Isto é, $t \in FOLLOW(X)$ se existe qualquer derivação contendo $X_t$. Isto pode ocorrer se a derivação $XYZ_t$ onde $Y$ e $Z$ ambos derivem $\varepsilon$.
\end{itemize}

\section*{Algoritmo}

O algoritmo de geração dos conjuntos first e follow tem como entrada o alfabeto da linguagem e a gramática a ser analisada.

O conjunto first, na primeira etapa, analisamos cada estado, varrendo as suas produções e verificando se o primeiro símbolo de cada produção é terminal, e o atribuímos ao conjunto first do estado em questão. Na segunda etapa, é verificado todos os símbolos não terminais da esquerda para a direita. Se existir um próximo símbolo, e esse for um símbolo terminal, então este símbolo entra no conjunto first do não-terminal sendo verificado. Se este próximo símbolo for um não-terminal, verifica-se se este estado possui $\varepsilon$ em seu conjunto first, então copia-se o first deste próximo estado, para o estado sendo verificado, e o próximo estado passa a ser verificado, seguindo os mesmos passos.

No conjunto follow, primeiramente, atribui-se ao follow do estado inicial da gramática, o símbolo $\$$ ($S' ::= S\$$).
Na primeira etapa do processamento, percorre-se todas as produções ($S$) da gramática, analisando os símbolos não-terminais. Para todos os não-terminais($S$), $FOLLOW(A) = First(B)$.
\\
\\
$S ::= ABS |aA $\\
$A ::= \varepsilon | a$\\
$B ::= Bb | cd$\\

Caso o próximo símbolo do não-terminal sendo verificado for um símbolo terminal, este é adicionado ao conjunto follow to não-terminal em questão.
Por exemplo, a produção $Bb$, e $B$ é o não-terminal sendo verificado, e $b$ é o próximo símbolo, então $b$ é adicionado ao conjunto follow de $B$.

Na segunda etapa, é verificado se o último caractere de cada produção é um não-terminal. Se for um não-terminal, faz-se o seguinte:\\
\\
$FOLLOW(A) = FOLLOW(A) \cup FOLLOW(S)$,\\

onde $S$ é o estado onde a produção esta sendo vericada.
Além disso verifica-se se este estado $A$ possui em seu conjunto first o $\varepsilon$, se sim passa a verificar o anterior, seguindo os mesmos conjuntos de passos descritos acima.

\section*{Considerações Finais}

Através do algoritmo o grupo conseguiu obter os conjuntos First e Follow das gramáticas livre de contexto. 
Podendo assim exercitar melhor nosso conhecimento deste tópico, tirando assim algumas dúvidas. 
\end{document}